\documentclass[a4paper,12pt]{article}
%%% Written by Matteo Guzzo
%%% A.D. MMXIV (2014)   
%%%
%%%
%\usepackage[T1]{fontenc}
 \usepackage{palatino}
 \linespread{1.05}
% \usepackage{hypen}`
 \usepackage{graphicx,amsmath,amssymb}

\def\be{\begin{equation}}
\def\ee{\end{equation}}
\def\bea{\begin{eqnarray}}
\def\eea{\end{eqnarray}}
\def\dd{\mbox{d}}

\renewcommand{\Re}{\,\mathrm{Re}}
\renewcommand{\Im}{\,\mathrm{Im}}

\title{Notes on the multipole model}
\author{Matteo Guzzo}
\date{\today}

%\pagestyle{empty}

\hyphenation{SPEC-TRO-SCO-PY}
\hyphenation{EN-ER-GY}

\begin{document}

\maketitle

\tableofcontents

\section{The multipole model}
Following Josh's prescription and notes, I resume here the main points of the 
multipole model and multipole curve fitting approach. 
Say we want to represent a function $ f ( \omega ) $ as a sum of delta functions, 
\emph{aka} poles. 
The original expression for the sum (correctly reported in Josh's paper 
[Kas J.J.\@ \emph{et al.}, PRB 76, 195116 (2007)]) reads
\be
 f(\omega) = \pi \sum_i^{N} g_i \omega_i^2 \delta( \omega^2 - \omega_i^2 ), 
 \label{eq:mpole2}
\ee
where we momentarily dispose of the $q$-dependence of the poles. 
This expression becomes exact for $ N \rightarrow \infty $, 
i.e. in practice the number of poles has to be converged. 
This expression can be rewritten (and has to be, when we want to calculate how to find 
the parameters $ g_i $ and $\omega_i $) in terms of simpler poles 
$ \delta( \omega - \omega_i ) $. 
In particular, it can be proved that, for $ \omega \geq 0 $, the following expression 
holds: 
\be
 f(\omega) = \frac{\pi}{2} \sum_i^{N} g_i \omega_i \delta( \omega - \omega_i ), 
 \label{eq:mpole1}
\ee
where  $ g_i $ and $\omega_i $ are the exact same parameters than in the previous case. 
The proof of this is shown below. 
This means that they can be calculated as before using the formula
\be
 \int_{\Delta_i} d \omega \; \omega^n f( \omega ) = \frac{\pi}{2} g_i \omega_i^{n+1} , 
 \label{eq:mpolemoments}
\ee
provided that the $ \Delta_i $ intervals are obtained correctly. 
In particular one can choose the intervals under the constrain to have the same area 
under the original curve $ f ( \omega ) $ within each interval. 
Then one shall use the relation above (Eq.\@ \eqref{eq:mpolemoments}) 
for first and first inverse moment ($ n = \pm 1 $) to calculate 
the parameters. 
This proposed here is just one possible way to do it. 
One could think of different choices which might be equally effective.

\subsection{The pedantic but necessary calculation} 
One starts from Eq.\@ \eqref{eq:mpole2} and uses the following relation: 
\be
 \delta[ g(x) ] = \sum_i \frac { \delta ( x - x_i ) }{ | g' ( x_i ) | }
\ee
which states that the delta function of a given function of $ x $ is equal to the sum 
of delta functions of $ x $ centered on the roots $ x_i $ of the $ g(x) $ function, 
divided by the absolute value of the first derivative of $ g $ evaluated at 
the poles. 
The easiest example for $ g(x) $ one can think of 
is a polinomial (like $  x^2 - x_i^2 $, where the roots are $ \pm x_i $). 
Applying this relation  to Eq.\@ \eqref{eq:mpole2} gives 
\be
 f(\omega) = \pi \sum_i^{N} g_i \omega_i^2 \delta( \omega^2 - \omega_i^2 ) 
 =  \frac{\pi}{2} \sum_i^{N} g_i \omega_i [ \delta( \omega - \omega_i ) + \delta( \omega + \omega_i ) ] , 
% \label{eq:mpole2}
\ee
which, for $ \omega \geq 0 $, gives Eq.\@ \eqref{eq:mpole1}. q.e.d. 

\section{Multipole fit in practice} 

There are surely a few ways to implement the multipole fit. 
Things are made a bit easier here, since we have the condition
$ \omega \geq 0 $. 
One has freedom on how to determine the intervals 
of the $N$ $ \Delta_i $, since they should coincide as 
$ N \rightarrow \infty $ (i.e. at convergence). 
As mentioned above, one possible choice is to have intervals such 
as the area under $ f(\omega) $ is constant. 
Another choice could be to have evenly-spaced intervals. 
Let us explore the latter first. 

\subsection{Evenly-spaced intervals}

Once one has determined the $ \Delta_i $, one can focus on a single 
interval at a time, using the moments' relation \eqref{eq:mpolemoments}. 
It is handy to first define a few quantities that will ease the notation. 
The 0-th moment is defined as $I_i$: 
\begin{equation}
	I_i = \frac{\pi}{2} g_i \omega_i  
  = \int_{\Delta_i} d \omega \; f( \omega ) . 
\end{equation}
The first moment is defined as $F_i$:
\begin{equation}
	F_i = \frac{\pi}{2} g_i \omega_i^2  
  = \int_{\Delta_i} d \omega \; \omega f( \omega )  
\end{equation}
and the first inverse moment is
\begin{equation}
	INV_i = \frac{\pi}{2} g_i 
	= \int_{\Delta_i} d \omega \; \frac1\omega f( \omega ) . 
\end{equation}

Using these formulas, one can first calculate $g_i$ as 
\begin{equation}
	g_i = \frac2{\pi} 
	\int_{\Delta_i} d \omega \; \frac1\omega f( \omega ) 
	=  \frac2{\pi} INV_i , 
\end{equation}
then one can calculate $\omega_i$ as 
\begin{equation}
	\omega_i = \sqrt{ \frac2{\pi g_i }
	\int_{\Delta_i} d \omega \; \omega f( \omega ) } = \sqrt{ \frac{F_i}{INV_i} } . 
\end{equation}
Now, the actual coefficients of the expansion are $ \lambda_i = \frac{\pi}{2} \omega_i g_i $, 
which can be calculated as follows: 
\begin{equation}
	\lambda_i =  \sqrt{ {F_i} \cdot {INV_i} } . 
\end{equation}


\end{document}
