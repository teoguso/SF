\documentclass[a4paper,12pt]{article}
%%% Written by Matteo Guzzo
%%% A.D. MMXIV (2014)   
%%%
%%%
%\usepackage[T1]{fontenc}
 \usepackage{palatino}
 \linespread{1.05}
\usepackage{mathpazo}
 
% \usepackage{hypen}`
 \usepackage{graphicx,amsmath,amssymb,amsfonts}

\def\be{\begin{equation}}
\def\ee{\end{equation}}
\def\bea{\begin{eqnarray}}
\def\eea{\end{eqnarray}}
\def\dd{\mbox{d}}

\renewcommand{\Re}{\,\mathrm{Re}}
\renewcommand{\Im}{\,\mathrm{Im}}

\title{Formulas in the code}
\author{Matteo Guzzo}
\date{\today}

%\pagestyle{empty}

\hyphenation{SPEC-TRO-SCO-PY}
\hyphenation{EN-ER-GY}

\begin{document}

\maketitle

\tableofcontents


\section{Exact renormalization}

Using the multipole fit for $W$, we realized that there is a divergence in the theory. 
However, it is possible that this divergence could be controlled if the theory is rewritten
in a certain way. 
Specifically, the renormalization factor should be kept with and expanded the same way as 
the satellites term, instead of being extracted after the explicit integration of $W(\tau)$. 
Let's see how. 
First, we take (from M.\@ Guzzo's thesis., eq.\@ 5.37) the expression for $G$ in the decoupling approximation. 
\be\label{eq:onelevelg_qp}
   {G} ( \tau ) =  {G}^{QP} ( \tau ) 
  \exp\left\{ \frac{i}{2\pi} \int\! d\omega\,
   {W} ( \omega )\left[ 
   \frac{ e^{i\omega \tau } - 1} 
     {\omega^2} \right]  \right\} ,
\ee
where $G^\textnormal{QP}$ is the ground-state Hartree Green's function
with QP energies and $W(\omega)$ is the Fourier transform of an element of
the screened Coulomb interaction. 

\subsection{Single-pole model}
We can proceed using the single plasmon-pole model for W, obtaining
\be
   {G} ( \tau ) =  i \theta ( -\tau ) 
  \exp\! \left( - i E^\textnormal{QP} \tau \right)
  \exp\left[ 
   \frac{\lambda}{\omega_p^2} \left( 
   e^{i\omega_p \tau } - 1 
      \right)  \right] ,
\ee
where $\lambda$ and $\omega_p$ are the strength and frequency of the plasmon, respectively. 
Expanding the  $e^{ i \omega_p \tau } - 1 $ term, we get
\begin{multline}
   {G} ( \tau ) =  i \theta ( -\tau ) 
  \exp\! \left( - i E^\textnormal{QP} \tau \right) \\
 \times \left\{ 1 + 
   \frac{\lambda}{\omega_p^2} \left(  e^{i\omega_p \tau } - 1 \right) 
  + \frac12 \left( \frac{\lambda}{\omega_p^2} \right)^2  \left( e^{i\omega_p \tau } - 1 \right)^2  \right. \\
  \left.  + \frac16 \left( \frac{\lambda}{\omega_p^2} \right)^3  \left( e^{i\omega_p \tau } - 1 \right)^3  
  + \ldots  \right\} .
\end{multline}
Let's define $a_p=\lambda/\omega_p^2$. Therefore
\begin{multline}
   {G} ( \tau ) =  i \theta ( -\tau ) 
  \exp\! \left( - i E^\textnormal{QP} \tau \right) \\
 \times \left\{ 1 + 
   a_p e^{i\omega_p \tau } - a_p 
  + \frac{a_p^2}2  e^{i 2\omega_p \tau }  + \frac{a_p^2}2 - a_p^2 e^{i \omega_p \tau }  \right. \\
  \left.  + \frac{a_p^3}6 e^{ i 3 \omega_p \tau } - \frac{a_p^3}6 
  + \frac{a_p^3}2 e^{ i \omega_p \tau } -  \frac{a_p^3}2 e^{ i 2 \omega_p \tau }
  + \ldots  \right\} ,
\end{multline}
which becomes
\begin{multline}
   {G} ( \tau ) =  i \theta ( -\tau ) 
  \exp\! \left( - i E^\textnormal{QP} \tau \right) \\
 \times \left\{ 1 - a_p +\frac{a_p^2}2 -\frac{a_p^3}6 
   + \left( a_p - a_p^2 +\frac{a_p}2 \right) e^{i\omega_p \tau }   \right. \\
  \left.  + \left( \frac{a_p^2}2 - \frac{a_p^3}2 \right) e^{i 2 \omega_p \tau }
  + \frac{a_p^3}6 e^{ i 3 \omega_p \tau } 
  + \ldots  \right\} .
\end{multline}
%
Now $G(\omega) = \int\!d\tau e^{i\omega\tau} G(\tau)$, so
\begin{multline}
   {G} ( \omega ) =  i \int\!d\tau \theta ( -\tau ) 
  \exp\! \left[ i (\omega - E^\textnormal{QP} ) \tau \right] \\
 \times \left\{ 1 - a_p +\frac{a_p^2}2 -\frac{a_p^3}6 
   + \left( a_p - a_p^2 +\frac{a_p}2 \right) e^{i\omega_p \tau }   \right. \\
  \left.  + \left( \frac{a_p^2}2 - \frac{a_p^3}2 \right) e^{i 2 \omega_p \tau }
  + \frac{a_p^3}6 e^{ i 3 \omega_p \tau } 
  + \ldots  \right\} \\
 = i\left[ \left( 1 - a_p +\frac{a_p^2}2 -\frac{a_p^3}6 \right) \delta (\omega - E^\textnormal{QP} ) \right. 
  + \left( a_p - a_p^2 +\frac{a_p}2 \right) \delta (\omega - E^\textnormal{QP} +\omega_p )  \\ 
  +  \left( \frac{a_p^2}2 - \frac{a_p^3}2 \right) \delta (\omega - E^\textnormal{QP} + 2 \omega_p ) 
  + \left. \frac{a_p^3}6 \delta (\omega - E^\textnormal{QP} + 3 \omega_p )+ \ldots  \right] . 
\end{multline}
Let's calculate $A(\omega) = 1/\pi \Im[G(\omega)]$: 
\begin{multline}
   A ( \omega ) =  \frac1\pi \left[ \left( 1 - a_p +\frac{a_p^2}2 -\frac{a_p^3}6 \right) \delta (\omega - E^\textnormal{QP} ) \right. \\
  + \left( a_p - a_p^2 +\frac{a_p}2 \right) \delta (\omega - E^\textnormal{QP} +\omega_p )  \\ 
  +  \left( \frac{a_p^2}2 - \frac{a_p^3}2 \right) \delta (\omega - E^\textnormal{QP} + 2 \omega_p ) \\
  + \left. \frac{a_p^3}6 \delta (\omega - E^\textnormal{QP} + 3 \omega_p )+ \ldots  \right] .
\end{multline}
Using a complex QP energy $E=\epsilon+\Gamma$, one gets
\begin{multline}
   A ( \omega ) =  \frac\Gamma\pi \left[ \left( 1 - a_p +\frac{a_p^2}2 -\frac{a_p^3}6 \right) \frac1{(\omega - \epsilon )^2 + \Gamma^2} \right. \\
  + \left( a_p - a_p^2 +\frac{a_p}2 \right) \frac1{(\omega - \epsilon +\omega_p )^2 + \Gamma^2}  \\ 
  +  \left( \frac{a_p^2}2 - \frac{a_p^3}2 \right) \frac1{(\omega - \epsilon + 2\omega_p )^2 + \Gamma^2}  \\
  + \left. \frac{a_p^3}6 \frac1{(\omega - \epsilon + 3\omega_p )^2 + \Gamma^2} + \ldots  \right] .
\end{multline}

\subsection{Multipole model}

Extending the formulation to a multipole model for $W$, with $N$ weights $\lambda_j$ and frequencies $\omega_j$, one obtains 
\be
   {G} ( \tau ) =  i \theta ( -\tau ) 
  \exp\! \left( - i E^\textnormal{QP} \tau \right)
  \exp\left[ \sum_j^N a_j \left(  e^{i\tilde\omega_j \tau } - 1 
      \right)  \right] ,
\ee
where $ a_j = \lambda_j/\omega_j^2 $. 
Expanding the exponential one gets
\begin{multline}
   {G} ( \tau ) =  i \theta ( -\tau ) 
  \exp\! \left( - i E^\textnormal{QP} \tau \right) \\
 \times \left\{ 1 + 
    \sum_j^N a_j \left(  e^{i\tilde\omega_j \tau } - 1  \right) 
  + \frac12 \left[ \sum_j^N a_j \left(  e^{i\tilde\omega_j \tau } - 1  \right)  \right]^2  \right. \\
  \left.  + \frac16  \left[ \sum_j^N a_j \left(  e^{i\tilde\omega_j \tau } - 1  \right)  \right]^3  
  + \ldots  \right\} .
\end{multline}


\section{The CRC approximation}

The formula is currently implemented to the third order, but omitting one term. 
Here's the formula in an easily readable version (including the term that will be neglected):
\begin{multline}
 A_{k,b}^\textnormal{unocc}(\omega) = Z_{k,b}^\textnormal{CRC} \Gamma_{k,b} \Biggl\{ 
  \sum_i^{N^<} \frac{\beta_{k,b}^>}{(\omega - \epsilon_{k,b} + \tilde\omega_i^<)^2 + \Gamma_{k,b}^2} \\
 + \frac12 \sum_{i\neq j}^{N^<} 
\frac{(\beta_{k,b}^>)^2}{(\omega - \epsilon_{k,b} + \tilde\omega_i^< + \tilde\omega_j^<)^2 + \Gamma_{k,b}^2}
 + \frac12 \sum_i^{N^<} \frac{(a_i + \beta_{k,b}^>)^2 - a_i^2}{(\omega - \epsilon_{k,b} + 2 \tilde\omega_i^<)^2 + \Gamma_{k,b}^2} \\
 + \frac16 \sum_i^{N^<} \frac{(a_i + \beta_{k,b}^>)^3 - a_i^3}{(\omega - \epsilon_{k,b} + 3 \tilde\omega_i^<)^2 + \Gamma_{k,b}^2} 
 + \frac12 \sum_{i\neq j}^{N^<} \frac{\bigl[(a_i + \beta_{k,b}^>)^2 - a_i^2\bigr] \beta_{k,b} }
  {(\omega - \epsilon_{k,b} + 2\tilde\omega_i^< + \tilde\omega_j^<)^2 + \Gamma_{k,b}^2} \\
 + \frac16 \sum_{i\neq j\neq k}^{N^<} 
\frac{(\beta_{k,b}^>)^3}{(\omega - \epsilon_{k,b} + \tilde\omega_i^< + \tilde\omega_j^< + \tilde\omega_k^<)^2 + \Gamma_{k,b}^2}
\Biggr\}
\end{multline}
and here it's how it's rewritten in the code. Again, the last term ($i\neq j\neq k$) is for now \emph{neglected}: 
\begin{multline}
 A_{k,b}^\textnormal{unocc}(\omega) = Z_{k,b}^\textnormal{CRC} \Gamma_{k,b}  \sum_i^{N^<} \Biggl( \beta_{k,b}^> \biggl\{
 \frac1{(\omega - \epsilon_{k,b} + \tilde\omega_i^<)^2 + \Gamma_{k,b}^2}  \\ 
  + \frac12 \sum_{j=0}^{i-1} \biggl[ \frac{(a_i + \beta_{k,b}^>)^2 - a_i^2 }
  {(\omega - \epsilon_{k,b} + 2 \tilde\omega_i^< + \tilde\omega_j^<)^2 + \Gamma_{k,b}^2} 
 + \frac{\beta_{k,b}^>}{(\omega - \epsilon_{k,b} + \tilde\omega_i^< + \tilde\omega_j^<)^2 + \Gamma_{k,b}^2}  \biggr] \\ 
  + \frac12 \sum_{j=i+1}^{N^<} \biggl[ \frac{(a_i + \beta_{k,b}^>)^2 - a_i^2 }
  {(\omega - \epsilon_{k,b} + 2 \tilde\omega_i^< + \tilde\omega_j^<)^2 + \Gamma_{k,b}^2} 
 + \frac{\beta_{k,b}^>}{(\omega - \epsilon_{k,b} + \tilde\omega_i^< + \tilde\omega_j^<)^2 + \Gamma_{k,b}^2}
 \biggr] \biggr\} \\
 + \frac12 \frac{(a_i + \beta_{k,b}^>)^2 - a_i^2}{(\omega - \epsilon_{k,b} + 2 \tilde\omega_i^<)^2 + \Gamma_{k,b}^2} 
 + \frac16 \frac{(a_i + \beta_{k,b}^>)^3 - a_i^3}{(\omega - \epsilon_{k,b} + 3 \tilde\omega_i^<)^2 + \Gamma_{k,b}^2} 
 \Biggl) . 
\end{multline}
The case $i=0$ is actually ill-defined, because of the $j= i-1$ sum. Therefore we will rewrite it as
\begin{multline}
 A_{k,b}^\textnormal{unocc}(\omega) = Z_{k,b}^\textnormal{CRC} \Gamma_{k,b} \Biggl[ 
   \Biggl( \beta_{k,b}^> \biggl\{
 \frac1{(\omega - \epsilon_{k,b} + \tilde\omega_0^<)^2 + \Gamma_{k,b}^2}  \\ 
  + \frac12 \sum_{j=1}^{N^<} \biggl[ \frac{(a_0 + \beta_{k,b}^>)^2 - a_0^2 }
  {(\omega - \epsilon_{k,b} + 2 \tilde\omega_0^< + \tilde\omega_j^<)^2 + \Gamma_{k,b}^2} 
 + \frac{\beta_{k,b}^>}{(\omega - \epsilon_{k,b} + \tilde\omega_0^< + \tilde\omega_j^<)^2 + \Gamma_{k,b}^2}
 \biggr] \biggr\} \\
 + \frac12 \frac{(a_0 + \beta_{k,b}^>)^2 - a_0^2}{(\omega - \epsilon_{k,b} + 2 \tilde\omega_0^<)^2 + \Gamma_{k,b}^2} 
 + \frac16 \frac{(a_0 + \beta_{k,b}^>)^3 - a_0^3}{(\omega - \epsilon_{k,b} + 3 \tilde\omega_0^<)^2 + \Gamma_{k,b}^2} 
 \Biggr) \\
 + \sum_{i=1}^{N^<} \Biggl( \beta_{k,b}^> \biggl\{
 \frac1{(\omega - \epsilon_{k,b} + \tilde\omega_i^<)^2 + \Gamma_{k,b}^2}  \\ 
  + \frac12 \sum_{j=0}^{i-1} \biggl[ \frac{(a_i + \beta_{k,b}^>)^2 - a_i^2 }
  {(\omega - \epsilon_{k,b} + 2 \tilde\omega_i^< + \tilde\omega_j^<)^2 + \Gamma_{k,b}^2} 
 + \frac{\beta_{k,b}^>}{(\omega - \epsilon_{k,b} + \tilde\omega_i^< + \tilde\omega_j^<)^2 + \Gamma_{k,b}^2}  \biggr] \\ 
  + \frac12 \sum_{j=i+1}^{N^<} \biggl[ \frac{(a_i + \beta_{k,b}^>)^2 - a_i^2 }
  {(\omega - \epsilon_{k,b} + 2 \tilde\omega_i^< + \tilde\omega_j^<)^2 + \Gamma_{k,b}^2} 
 + \frac{\beta_{k,b}^>}{(\omega - \epsilon_{k,b} + \tilde\omega_i^< + \tilde\omega_j^<)^2 + \Gamma_{k,b}^2}
 \biggr] \biggr\} \\
 + \frac12 \frac{(a_i + \beta_{k,b}^>)^2 - a_i^2}{(\omega - \epsilon_{k,b} + 2 \tilde\omega_i^<)^2 + \Gamma_{k,b}^2} 
 + \frac16 \frac{(a_i + \beta_{k,b}^>)^3 - a_i^3}{(\omega - \epsilon_{k,b} + 3 \tilde\omega_i^<)^2 + \Gamma_{k,b}^2} 
 \Biggr) \Biggr]
\end{multline}


This is how the normalization factor is defined: 
\begin{equation}
  Z_{k,b}^\textnormal{CRC} = Z_{k,b}^\textnormal{TOC} e^{ - \beta_{k,b}^> } =  e^{ - \sum_i a_i }  e^{ - \beta_{k,b}^> } 
\end{equation}
and how the occupied spectral function is defined:
\begin{equation}
  A_{k,b}^\textnormal{occ}(\omega) = e^{ - \beta_{k,b}^> } A_{k,b}^\textnormal{TOC}(\omega) .
\end{equation}
The TOC spectral function is the old version from PRL 107. 
The total $A^\textnormal{CRC}(\omega)$ is 
\begin{equation}
   A_\textnormal{TOT}^\textnormal{CRC}(\omega) = \sum_k^\textnormal{occ} \left[ A_{k,b}^\textnormal{occ}(\omega) + A_{k,b}^\textnormal{unocc}(\omega) \right]
\end{equation}

\subsection{Parameter $\beta$ for the unoccupied part of the self-energy}

When $\beta_k$ is calculated using only one pole, i.e. $N^>=1$, (other cases were not tested) it has to be normalized with the number of poles $N^<$: 
\begin{equation}
	\beta_{k,b}^>  = \frac1{N^<} \sum_i^{N^>} \frac{\lambda_{i,(k,b)}^>}{(\tilde\omega_{i,(k,b)}^>)^2} = \frac1{N^<} \sum_i b_{i,(k,b)}.
\end{equation}
However, it is not divided by $N^<$ when used in the definition of $Z^\textnormal{CRC}$. 
The consistency of this expression should be clarified in Sky's PhD thesis. 

\end{document}
