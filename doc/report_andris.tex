\documentclass[a4paper,12pt]{article}
%%% Written by Matteo Guzzo
%%% A.D. MMXIV (2014)   
%%%
%%%
%\usepackage[T1]{fontenc}
 \usepackage{palatino}
 \linespread{1.05}
\usepackage{mathpazo}
 
% \usepackage{hypen}`
 \usepackage{graphicx,amsmath,amssymb,amsfonts}

\def\be{\begin{equation}}
\def\ee{\end{equation}}
\def\bea{\begin{eqnarray}}
\def\eea{\end{eqnarray}}
\def\dd{\mbox{d}}

\renewcommand{\Re}{\,\mathrm{Re}}
\renewcommand{\Im}{\,\mathrm{Im}}

\title{Update on the cumulant, renormalization and the use of $\Gamma$}
\author{Matteo Guzzo}
\date{\today}

%\pagestyle{empty}

\hyphenation{SPEC-TRO-SCO-PY}
\hyphenation{EN-ER-GY}

\begin{document}

\maketitle

\tableofcontents

\section{The misleading misunderstanding in Guzzo's thesis}

After discussing with Andris about my thesis, we manage to uncover a couple weak 
points in the theory, and a couple mistakes as well. 
We can focus on equations 5.26 and 5.29. 
Equation 5.26 is merely the $G(\tau)$ solution of the linearized decoupled EOM for $G$, only 
rewritten using the FT of ${W}(\tau)$  
and it reads
\be
 G(\tau) = G^H (\tau) \exp \left[ \frac{i}{2\pi}  \int \! d\omega \frac{ W (\omega) }{ \omega^2 } 
   \left( e^{i\omega\tau} - i \omega \tau - 1 \right) \right] ,
\ee 
where $G^H$ is the Linear-response ground-state Hartree Green's function and 
$W(\omega)$ the FT of a matrix element of the screened Coulomb interaction. 
This equation, while being a correct result, lacked an additional comment: 
if one assumes that $W(\omega)$ is a symmetric function (i.e.\@ an even function of $\omega$) 
--- and this seems to be a reasonable assumption for any kind of $W$ --- 
the equation further simplifies, since the middle term in parenthesis ($i\omega\tau$)
produces the integral 
\be
  \int \! d\omega \frac{ W (\omega) }{ \omega } 
\ee 
(times unimportant factors) that is identical to zero. 
This is verified because the product of an even function with an odd function is an odd function, 
and the integral over $R$ of an odd function is always zero. 
The formula should therefore be written as 
\be
 G(\tau) = G^H (\tau) \exp \left[ \frac{i}{2\pi}  \int \! d\omega \frac{ W (\omega) }{ \omega^2 } 
   \left( e^{i\omega\tau} - 1 \right) \right] . 
\ee 
There is nothing wrong so far, but it is a good thing to know. 
Let's now focus on eq.\@ 5.29. 
The equation reads
\be
 G(\tau) = G^H (\tau) \exp \left[ \frac{\lambda}{\omega_p^2}  
   \left( e^{i\omega_p\tau} - i \omega_p \tau - 1 \right) \right] . 
\ee
This is, again, a $G(\tau)$ but for a particular case: one can obtain this solution when a 
single-plasmon-pole model is used for $W(\tau)$, where $\lambda$ and $\omega_p$ are 
the weight and frequency of the plasmon, respectively. 
\textbf{As one can see now, the second term in this formula ($-i\omega_p\tau$) is \emph{not} 
connected to the second term in the first formula ($-i\omega\tau$) as I originally thought. 
In fact, one can obtain eq.\@ 4 by inserting the plasmon-pole $W(\omega)$ in eq.\@ 3 and using the
residuals theorem in the complex plane.}

This result can be easily generalised to the case of many poles for $W$, which is exact for
an infinite number of poles $N\rightarrow\infty$.The result is equivalent and reads
\be
 G(\tau) = G^H (\tau) \exp \left[ \sum_j^N \frac{\lambda_j}{\tilde\omega_j^2}  
   \left( e^{i\tilde\omega_j\tau} - i \tilde\omega_j \tau - 1 \right) \right] . 
\ee
Single- and multipole cases are completely equivalent, with only a slight complication in the 
case of $N$ poles. 
It can be shown that the second term in the exponent $i \sum_j^N \frac{\lambda_j}{\tilde\omega_j} \tau $
is actually the $G_0 W_0 $ self-energy at the QP (Hartree???) energy. 


\end{document}
